\subsubsection{LGV引理}

\paragraph{定义}

$\omega(P)$ 表示 $P$ 这条路径上所有边的边权之积。(路径计数时,可以将边权都设为 $1$)(事实上,边权可以为生成函数)

$e(u, v)$ 表示 $u$ 到 $v$ 的 \textbf{每一条} 路径 $P$ 的 $\omega(P)$ 之和,即 $e(u, v)=\sum\limits_{P:u\rightarrow v}\omega(P)$。

起点集合 $A$,是有向无环图点集的一个子集,大小为 $n$。

终点集合 $B$,也是有向无环图点集的一个子集,大小也为 $n$。

一组 $A\rightarrow B$ 的不相交路径 $S$:$S_i$ 是一条从 $A_i$ 到 $B_{\sigma(S)_i}$ 的路径($\sigma(S)$ 是一个排列),对于任何 $i\ne j$,$S_i$ 和 $S_j$ 没有公共顶点。

$t(\sigma)$ 表示排列 $\sigma$ 的逆序对个数。

\paragraph{引理}

$$
M = \begin{bmatrix}e(A_1,B_1)&e(A_1,B_2)&\cdots&e(A_1,B_n)\\
e(A_2,B_1)&e(A_2,B_2)&\cdots&e(A_2,B_n)\\
\vdots&\vdots&\ddots&\vdots\\
e(A_n,B_1)&e(A_n,B_2)&\cdots&e(A_n,B_n)\end{bmatrix}
$$

$$
\det(M)=\sum\limits_{S:A\rightarrow B}(-1)^{t(\sigma(S))}\prod\limits_{i=1}^n \omega(S_i)
$$

其中 $\sum\limits_{S:A\rightarrow B}$ 表示满足上文要求的 $A\rightarrow B$ 的每一组不相交路径 $S$。