建图

\begin{minted}{cpp}
u32 n;
std::string s;
std::cin >> n;
tifa_libs::str::ex_suffix_automaton sam;
flt_(u32, i, 0, n) {
  std::cin >> s;
  u32 last = 0;
  for (auto c : s)
  last = sam.extend(last, u32(c - 'a'));
}
\end{minted}

代码对此无体现\\
\\

多个字符串间的最长公共子串, 对每个状态建立一个集合 \verb|pos|, 每次插入第 \verb|i| 个串的一个字符后

\verb|st.pos[last].push_back(i)|, 则对于每个状态 \verb|u|

\[
    \forall a,b \in \texttt{st[u].pos}~\text{then}~\texttt{lcs(a, b)}~\text{is}~\texttt{str(u)}
\]

\[
    \forall a\in \texttt{st[u].pos}, b\in \texttt{st[st[u].link].pos}~\text{then}~\texttt{lcs(a, b)} ~\text{in}~ \texttt{str(u, st[u].link)}
\]

其中 \verb|lcs(a,b)| 表示串 \verb|a|, \verb|b| 的最长公共子串, \verb|str(u)| 表示状态 \verb|u| 表示的串, \verb|str(u, v)| 表示在后缀链接中 \verb|u|, \verb|v| 这条边的表示状态表示的串

然后按照 \verb|len| 递减的顺序遍历, 通过后缀链接将当前状态的 \verb|pos| 与其他状态的合并, 遍历所有的状态, 找到一个 \verb|len| 最大且满足当前 \verb|pos| 的 \verb|size| 为 \verb|n| 的状态的 \verb|len| 即为解

两两之间的最长公共子串也可这么求

\paragraph{复杂度}

\(O(|\sigma| \log|\sigma|)\)
