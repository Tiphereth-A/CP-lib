\subsection{线性代数}

\subsubsection{行列式}

\begin{equation}
    \operatorname{det}(A):=\sum_{\sigma\in S_n}\prod_{i=1}^n \operatorname{sgn}(\sigma) a_{i,\sigma(i)}
\end{equation}

\subsubsection{积和式}

\begin{equation}
    \operatorname{per}(A):=\sum_{\sigma\in S_n}\prod_{i=1}^n a_{i,\sigma(i)}
\end{equation}

可用于完美匹配中

\paragraph{性质}

\begin{equation}
    \operatorname{per}(A)=\operatorname{haf}\begin{bmatrix}
            & A \\
        A^T & 
    \end{bmatrix}
\end{equation}

\(\operatorname{haf}(A)\) 的定义参见 \fullref{sec:图的完美匹配计数}

\subsubsection{邻接矩阵行列式的意义}

在无向图中取若干个环, 一种取法权值就是边权的乘积, 对行列式的贡献是 \((-1)^{e}\), 其中 \(e\) 是偶环的个数


\subsubsection{LGV 引理}

\paragraph{定义}

\(\omega(P)\) 表示 \(P\) 这条路径上所有边的边权之积. (路径计数时, 可以将边权都设为 \(1\)) (事实上, 边权可以为生成函数)

\(e(u, v)\) 表示 \(u\) 到 \(v\) 的 \textbf{每一条} 路径 \(P\) 的 \(\omega(P)\) 之和, 即 \(e(u, v)=\sum\limits_{P:u\rightarrow v}\omega(P)\).

起点集合 \(A\), 是有向无环图点集的一个子集, 大小为 \(n\).

终点集合 \(B\), 也是有向无环图点集的一个子集, 大小也为 \(n\).

一组 \(A\rightarrow B\) 的不相交路径 \(S\): \(S_i\) 是一条从 \(A_i\) 到 \(B_{\sigma(S)_i}\) 的路径(\(\sigma(S)\) 是一个排列), 对于任何 \(i\ne j\), \(S_i\) 和 \(S_j\) 没有公共顶点.

\(t(\sigma)\) 表示排列 \(\sigma\) 的逆序对个数.

\paragraph{引理}

\[
    M = \begin{bmatrix}
        e(A_1,B_1) & e(A_1,B_2) & \cdots & e(A_1,B_n) \\
        e(A_2,B_1) & e(A_2,B_2) & \cdots & e(A_2,B_n) \\
        \vdots     & \vdots     & \ddots & \vdots     \\
        e(A_n,B_1) & e(A_n,B_2) & \cdots & e(A_n,B_n)
    \end{bmatrix}
\]

\[
    \det(M)=\sum\limits_{S:A\rightarrow B}(-1)^{t(\sigma(S))}\prod\limits_{i=1}^n \omega(S_i)
\]

其中 \(\sum\limits_{S:A\rightarrow B}\) 表示满足上文要求的 \(A\rightarrow B\) 的每一组不相交路径 \(S\).

\subsection{组合数学}

\subsubsection{组合数相关公式}

\begin{equation}
    \sum_{i=0}^n(-1)^i\binom{n}{i}=[n=0]
\end{equation}

\begin{equation}
    \sum_{i=0}^m \binom{n}{i}\binom{m}{m-i} = \binom{m+n}{m}\qquad(n \geq m)
\end{equation}

\begin{equation}
    \sum_{i=0}^n\binom{n}{i}^2=\binom{2n}{n}
\end{equation}

\begin{equation}
    \sum_{i=0}^ni\binom{n}{i}=n2^{n-1}
\end{equation}

\begin{equation}
    \sum_{i=0}^ni^2\binom{n}{i}=n(n+1)2^{n-2}
\end{equation}

\begin{equation}
    \sum_{i=0}^n\binom{n-i}{i}=F_{n+1}
\end{equation}

其中 \(F\) 是 Fibonacci 数列

\begin{equation}
    g_n = \sum_{i = 0}^{n} \binom{n}{i} f_i \iff f_n = \sum_{i = 0}^{n} \binom{n}{i} (-1)^{n-i} g_i
\end{equation}

\subsubsection{Stirling 数相关公式}

\begin{equation}
    \sum_{n\geq k}\frac{x^n}{n!}\stirlingI{n}{k}=\frac{\left(-\ln(1-x)\right)^k}{k!}
\end{equation}

\begin{equation}
    \sum_{n\geq k}\frac{x^n}{n!}\stirlingII{n}{k}=\frac{\left(\mathrm{e}^x-1\right)^k}{k!}
\end{equation}

\begin{equation}
    \sum_{n\geq k}x^n\stirlingII{n}{k}=\prod_{i=1}^k\frac{x}{1-\mathrm{i}x}
\end{equation}

\begin{equation}
    f_n=\sum_{k=0}^n\stirlingI{n}{k}g_k \iff g_n=\sum_{k=0}^n\stirlingII{n}{k}(-1)^{n-k}f_k
\end{equation}

\begin{equation}
    \stirlingI{n}{k}=\frac{n!}{k!}\frac{k}{n}\left[x^{n-k}\right]\left(\frac{x}{\mathrm{e}^x-1}\right)^n
\end{equation}

\begin{equation}
    \stirlingII{n}{k}=\frac{n!}{k!}\frac{k}{n}\left[x^{n-k}\right]\left(\frac{x}{\ln(1+x)}\right)^n
\end{equation}

注意: \eqref{cs:formula:s-right} 是对的, \fullref{sec:theoretical-computer-science-cheat-sheet} 的第 43 个是错的

\begin{equation}
    \label{cs:formula:s-right}
    \stirlingI{n+m+1}{m}=\sum_{k=0}^m(n+k)\stirlingI{k+n}{k}
\end{equation}

\begin{equation}
    n^{\underline{n-m}}[n\geq m]=\sum_{k=0}^n\stirlingI{n+1}{k+1}\stirlingII{k}{m}(-1)^{m-k}
\end{equation}

\begin{equation}
    \stirlingII{-n+m}{-n}=\sum_k\binom{m-n}{m+k}\binom{m+n}{n+k}\stirlingI{-k}{-k-m}
\end{equation}

\subsubsection{Burnside 引理}

令 \(G\) 为置换群, \(X\) 为一集合, 则

\[
    |X/G|=\frac{1}{|G|}\sum_{g\in G}\left|X^g\right|
\]

其中 \(X^g:=\{x\in X:gx=x\}\) 表示 \(X\) 中关于 \(g\) 的稳定化子

推论(P\'olya 定理):

\[
    \left|Y^X/G\right|=\frac{1}{|G|}\sum_{g\in G} |Y|^{c(g)}
\]

其中 \(c(g)\) 为置换 \(g\) 能拆成的不相交循环置换的数量

推论: 设 \(f(k)\) 为对长为 \(k\) 的序列按固定顺序计数的结果, 此时可令 \(G=\mathbb{Z}_n\), 从而

\[
    g(n)=\frac{1}{n}\sum_{k=0}^{n-1}f(\gcd(n,k))=\frac{1}{n} \{f*\varphi\}(n)
\]

其中 \(*\) 为 Dirichlet 卷积

如: 长为 \(n\) 的项链, 珠子可以染 \(m\) 种颜色, 则总方案数为

\[
    \frac{1}{n}\sum_{k=0}^{n-1} m^{(n,k)}
\]

\subsubsection{自然数幂次和关于次数的EGF}

\begin{equation}
    \begin{aligned} 
        F(x)= & \sum_{k=0}^\infty \frac{\sum_{i=0}^n i^k}{k!}x^k \\ 
        =     & \sum_{i=0}^n \mathrm{e}^{ix}                     \\
        =     & \frac{\mathrm{e}^{(n+1)x-1}}{\mathrm{e}^x-1}
    \end{aligned}
\end{equation}


\subsection{初等数论}

\subsubsection{数论函数相关}

\begin{multicols}{3}
    \(\displaystyle \sum_{\delta\mid n}\varphi(\delta)d\left(\frac{n}{\delta}\right) = \sigma(n)\) \\
    \(\displaystyle \sum_{\delta\mid n}\left|\mu(\delta)\right| = 2^{\omega(n)}\) \\
    \(\displaystyle \sum_{\delta\mid n}2^{\omega(\delta)} = d(n^2)\) \\
    \(\displaystyle \sum_{\delta\mid n}d(\delta^2) = d^2(n)\) \\
    \(\displaystyle \sum_{\delta\mid n}d\left(\frac{n}{\delta}\right)2^{\omega(\delta)} = d^2(n)\) \\
    \(\displaystyle \sum_{\delta\mid n}\frac{\mu(\delta)}{\delta} = \frac{\varphi(n)}{n}\) \\
    \(\displaystyle \sum_{\delta\mid n}\frac{\mu(\delta)}{\varphi(\delta)} = d(n)\) \\
    \(\displaystyle \sum_{\delta\mid n}\frac{\mu^2(\delta)}{\varphi(\delta)} = \frac{n}{\varphi(n)}\)
\end{multicols}

\begin{equation}
    \sum_{i=1}^n\sum_{j=1}^m(i,j)^k=\sum_{D=1}^{\min\{n,m\}}\left\lfloor\frac{n}{D}\right\rfloor\left\lfloor\frac{m}{D}\right\rfloor\{\operatorname{id}_k*\mu\}(D)
\end{equation}

\begin{equation}
    \{\operatorname{id}_k*\mu\}(p^s)=p^{ks}-p^{k(s-1)}
\end{equation}

\begin{equation}
    \sum_{i = 1} ^ n \left[(i, n) = 1\right] i = n \frac {\varphi(n) + e(n)} 2
\end{equation}

\begin{equation}
    \sum_{i = 1} ^ n \sum_{j = 1} ^ i \left[(i, j) = d\right] = S_\varphi \left( \left\lfloor \frac n d \right\rfloor \right)
\end{equation}

\begin{equation}
    \sum_{i = 1} ^ n \sum_{j = 1} ^ m \left[(i, j) = d\right] = \sum_{d \mid  k} \mu\left( \frac k d \right) \left\lfloor \frac n k \right\rfloor \left\lfloor \frac m k \right\rfloor
\end{equation}

\begin{equation}
    \sum_{i = 1} ^ n f(i) \sum_{j = 1} ^ {\left\lfloor \frac n i \right\rfloor} g(j) = \sum_{i = 1} ^ n g(i) \sum_{j = 1} ^ {\left\lfloor \frac n i \right\rfloor} f(j)
\end{equation}

\begin{equation}
    \mu^2(n) = \sum_{d^2 \mid n} \mu (d)
\end{equation}

\begin{equation}
    \prod_{k=1,(k, m) = 1}^{m} k \equiv
    \begin{cases}
        -1 \pmod m, & m = 4, p^q, 2p^q \\
        1 \pmod m,  & \text{otherwise} \\
    \end{cases}
\end{equation}

\begin{equation}
    \sigma_k(n) = \sum_{d\mid n}d^k = \prod_{i=1}^{\omega(n)}\frac{p_i^{(a_i+1)k}-1}{p_i^k-1}
\end{equation}

\begin{eqnarray}
    &&J_k(n) = n^k\prod_{p\mid n}\left(1-\frac{1}{p^k}\right) \\
    &&\sum_{\delta\mid n}J_k(\delta) = n^k \\
    &&\sum_{\delta\mid n}\delta^sJ_r(\delta)J_s\left(\frac{n}{\delta}\right) = J_{r+s}(n)
\end{eqnarray}

其中 \(J_k(n)\) 是不大于 \(n\) 的正整数构成的 \(k\) 元组中, 与 \(n\) 构成互素 \((k + 1)\) 元组的个数

\begin{equation}
    \sum_{i=1}^{n} \sum_{j=1}^{n} [(i,j)=1]ij =  \sum_{i=1}^{n} i^2\varphi(i)
\end{equation}

\begin{equation}
    n\mid \varphi(a^n-1)
\end{equation}

\begin{equation}
    \sum_{\substack{1 \leq k \leq n,~(k, n) = 1}}f(\gcd(k-1, n)) = \varphi(n)\sum_{d\mid n}\frac{(\mu*f)(d)}{\varphi(d)}
\end{equation}

\begin{equation}
    \varphi([m, n])\varphi(\gcd(m,n)) = \varphi(m)\varphi(n)
\end{equation}

\begin{equation}
    \sum_{\delta\mid n}d^3(\delta) = \left(\sum_{\delta\mid n}d(\delta)\right)^2
\end{equation}

\begin{equation}
    d(uv) = \sum_{\delta\mid (u, v)}\mu(\delta)d\left(\frac{u}{\delta}\right)d\left(\frac{v}{\delta}\right)
\end{equation}

\begin{equation}
    \sigma_k(u)\sigma_k(v) = \sum_{\delta\mid (u, v)}\delta^k\sigma_k(\frac{uv}{\delta^2})
\end{equation}

\begin{equation}
    \mu(n) = \sum_{k=1}^n[(k, n)=1]\cos{2\pi \frac{k}{n}}
\end{equation}

\begin{equation}
    \varphi(n) = \sum_{k=1}^n[(k, n)=1] = \sum_{k=1}^n(k, n)\cos{2\pi \frac{k}{n}}
\end{equation}

\begin{equation}
    \begin{cases}
        S(n) = \sum_{k=1}^n(f * g)(k) \\
        \sum_{k=1}^nS(\lfloor \frac n k \rfloor) = \sum_{i=1}^nf(i)\sum_{j=1}^{\lfloor n/i \rfloor}(g * 1)(j)
    \end{cases}
\end{equation}

\begin{equation}
    \begin{cases}
        S(n) = \sum_{k=1}^n(f \cdot g)(k), g \text{ completely multiplicative} \\
        \sum_{k=1}^nS\left(\left\lfloor \frac n k \right\rfloor\right)g(k) = \sum_{k=1}^n(f * 1)(k)g(k)
    \end{cases}
\end{equation}

\subsubsection{升幂引理}

对 \(n>0\), \(p\in\mathbb{P}\), \(x,y\in\mathbb{Z}\), \(p\nmid x,y\) 且 \(p\mid x-y\):

\begin{itemize}
    \item 对奇素数 \(p\) 或满足 \(4\mid x-y\) 的偶素数 \(p\), 有 \(v_p\left(x^n-y^n\right)=v_p(x-y)+v_p(n)\)
    \item \(p=2\) 且 \(2\mid n\), 有 \(v_p\left(x^n-y^n\right)=v_p\left(\left(x^2\right)^{n/2}-\left(y^2\right)^{n/2}\right)\)
\end{itemize}

\subsubsection{Pythagorean 三元组}

\[
    \begin{array}{ccc}
        a=k\cdot\left(m^2-n^2\right) & b=k\cdot\left(2mn\right) & c=k\cdot\left(m^2+n^2\right)
    \end{array}
\]

其中 \(0<n<m\), \(k>0\), \(m\perp n\), 且 \(m\), \(n\) 中恰有一个奇数

\subsection{概率论与统计}

\begin{equation}
    D(X)=E(X-E(X))^2=E\left(X^2\right)-(E(X))^2
\end{equation}

\begin{equation}
    D(X+Y)=D(X)+D(Y)D(aX)=a^2D(X)
\end{equation}

\begin{equation}
    E[x]=\sum_{i=1}^{\infty}P(X\geq i)
\end{equation}

\(m\) 个数的方差: \(\displaystyle s^2=\frac{\sum_{i=1}^m x_i^2}m-\overline x^2\)

\subsection{kMAX-MIN 反演}

\begin{equation}
    \operatorname{k-max} S=\sum_{T\subset S, T\neq \emptyset}(-1)^{|T|-k}\binom{|T|-1}{k-1}\min T
\end{equation}

\begin{equation}
    \max S=\sum_{T\subset S, T\neq \emptyset}(-1)^{|T|-1}\min T
\end{equation}

\subsection{杂项}

\begin{equation}
    a>b,(a,b)=1 \implies (a^m-b^m,a^n-b^n)=a^{(m,n)}-b^{(m,n)}
\end{equation}

\begin{equation}
    \lfloor x\rfloor=\begin{cases}
        \displaystyle x+\frac{\arctan\cot\pi x}{\pi}-\frac{1}{2}, & x\notin\mathrm{Z} \\
        x,                                                        & x\in\mathrm{Z}
    \end{cases}
\end{equation}

\begin{equation}
    \mathrm{e}(\mathrm{e}+\pi)^\mathrm{e}(2e + \pi)^\mathrm{e}+\frac{\mathrm{e}}{(\pi^\mathrm{e}-\mathrm{e})(\mathrm{e}+\mathrm{e}^\pi-\pi^\mathrm{e})}\approx 114514.1919810
\end{equation}

\begin{equation}
    \begin{aligned}
        \sum_{i=0}^n(a+di)r^i & =\begin{cases}
                                     na+\dfrac{n(n-1)}{2}d,                                        & r=1    \\
                                     \dfrac{a(1-r^n)-(n-1)r^nd}{1-r}+\dfrac{1-r^{n-1}}{(1-r)^2}rd, & r\ne 1 \\
                                 \end{cases} \\
                              & \to\begin{cases}
                                       \infty,                             & |r|\geq 1 \\
                                       \dfrac{a}{1-r}+\dfrac{rd}{(1-r)^2}, & |r|<1
                                   \end{cases}\qquad(n\to+\infty)
    \end{aligned}
\end{equation}

\begin{equation}
    S_j = \sum_{k=1}^nx_k^j 
\end{equation}

\begin{eqnarray}
    &h_m = \sum_{1\leq j_1 < \cdots < j_m \leq n} x_{j_1}\cdots x_{j_m}\\
    &H_m = \sum_{1\leq j_1 \leq \cdots \leq j_m \leq n} x_{j_1}\cdots x_{j_m}
\end{eqnarray}

\begin{eqnarray}
    h_n = \frac{1}{n}\sum_{k=1}^n(-1)^{k+1}S_kh_{n-k} 
    H_n = \frac{1}{n}\sum_{k=1}^nS_kH_{n-k} 
\end{eqnarray}

\begin{equation}
    \sum_{k=0}^nkc^k = \frac{nc^{n+2}-(n+1)c^{n+1}+c}{(c-1)^2}
\end{equation}

\begin{equation}
    \sum_{i=1}^n\frac{1}{n}\approx\ln\left(n + \frac 1 2\right) + \frac{1}{24(n+0.5)^2}+\Gamma,\qquad \Gamma\approx0.5772156649015328606065
\end{equation}

\begin{equation}
    n! = \sqrt{2\pi n}\left(\frac{n}{\mathrm{e}}\right)^n\left(1+\frac{1}{12n}+\frac{1}{288n^2}+O\left(\frac{1}{n^3}\right)\right)
\end{equation}

\begin{equation}
    (\max-\min){\{x_a-x_b, y_a-y_b, z_a-z_b\}} = \frac{1}{2}\sum_{cyc}\left| (x_a-y_a)-(x_b-y_b) \right|
\end{equation}

\begin{equation}
    \sum_{cyc}(a+b) = \frac{(a+b+c)^3 - a^3 - b^3 - c^3}{3}
\end{equation}

划分问题: \(n\) 个 \(k-1\) 维向量最多把 \(k\) 维空间分为 \(\sum_{i=0}^{k}\binom{n}{i}\) 份.

设 \(U\), \(V\), \(W\), \(u\), \(v\), \(w\) 是四面体的边长 (\(U\), \(V\), \(W\) 构成三角形; \(u\) 是 \(U\) 的对边), 则该四面体的面积为

\begin{equation}
    V = \frac{\sqrt{ 4u^2v^2w^2 - \sum_{cyc}{u^2(v^2+w^2-U^2)^2} + \prod_{cyc}{(v^2+w^2-U^2)} }}{12}
\end{equation}
