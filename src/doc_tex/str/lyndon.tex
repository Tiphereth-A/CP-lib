\paragraph{最小表示法}

当字符串 \(S\) 中可以选定一个位置 \(i\) 满足 \(S[i\cdots n]+S[1\cdots i-1]=T\), 则称 \(S\) 与 \(T\) 循环同构

字符串 \(S\) 的最小表示为与 \(S\) 循环同构的所有字符串中字典序最小的字符串

\paragraph{Lyndon 分解}~\\

Lyndon 串: 对于字符串 \(s\), 如果 \(s\) 的字典序严格小于 \(s\) 的所有后缀的字典序, 我们称 \(s\) 是简单串, 或者 \textbf{Lyndon 串}

举一些例子, \verb|a|, \verb|b|, \verb|ab|, \verb|aab|, \verb|abb|, \verb|ababb|, \verb|abcd| 都是 Lyndon 串

当且仅当 \(s\) 的字典序严格小于它的所有非平凡 (非空且不同于自身) 循环同构串时, \(s\) 才是 Lyndon 串

Lyndon 分解: 串 \(s\) 的 Lyndon 分解记为 \(s=w_1w_2\cdots w_k\), 其中所有 \(w_i\) 为简单串, 且其字典序按非严格单减排序, 即 \(w_1\ge w_2\ge\cdots\ge w_k\). 可以发现, 这样的分解存在且唯一

对于长度为 \(n\) 的串 \(s\), 我们可以通过 Lyndon 分解寻找该串的最小表示法

我们构建串 \(ss\) 的 Lyndon 分解, 然后寻找这个分解中的一个 Lyndon 串 \(t\), 使得它的起点小于 \(n\) 且终点大于等于 \(n\)

可用 Lyndon 分解的性质证明: 子串 \(t\) 的首字符就是 \(s\) 最小表示法的首字符, 即沿着 \(t\) 的开头往后 \(n\) 个字符组成的串就是 \(s\) 的最小表示法

于是我们在分解的过程中记录每一次的近似 Lyndon 串的开头即可

\begin{minted}{cpp}
// smallest_cyclic_string
string min_cyclic_string(string s) {
  s += s;
  int n = s.size();
  int i = 0, ans = 0;
  while (i < n / 2) {
    ans = i;
    int j = i + 1, k = i;
    while (j < n && s[k] <= s[j]) {
      if (s[k] < s[j])
        k = i;
      else
        k++;
      j++;
    }
    while (i <= k) i += j - k;
  }
  return s.substr(ans, n / 2);
}
\end{minted}

\paragraph{复杂度}

\(O(|s|)\)

\paragraph{参考链接}

\qrcode{https://oi-wiki.org/string/minimal-string/}
\qrcode{https://oi-wiki.org/string/lyndon/}
