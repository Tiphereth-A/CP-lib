区间加,区间和,区间最值

\begin{tabular}{ll}
    \hline
    代码                                  & 备注                       \\
    \hline
    \verb|class T|                        & 维护的数据                 \\
    \verb|T (*op)(T, T)|                  & 维护的操作                 \\
    \verb|T (*e)()|                       & T的幺元                    \\
    \verb|class F|                        & 修改的类型(会用于lazy标记) \\
    \verb|T (*mapping)(F, T) f(x)|        & 即修改怎么应用于数据       \\
    \verb|F (*composition)(F, F) f(g(x))| & 即修改的复合               \\
    \verb|F (*id)() id(x) = x|            & 即恒等映射                 \\
    \hline
\end{tabular}

\paragraph{例子}

\textbf{没有区间操作}

\verb|F = T, maaping = op, id = composition = e|

\textbf{维护}: 区间min; \textbf{操作}: 区间加

\verb|T = {int _min}| 区间和, 长度

\begin{minted}{cpp}
op(T a, T b) = return T{min(a._min, b._min)}
e() = return T{INT_MAX}
F() = {int add}
mapping(F f, T a) = return T{a.s + f.add * a.l, a.l}
composition(F f, F g) = return F{f.add + g.add}
id() = return F{0}
\end{minted}

\textbf{维护}: 区间和; \textbf{操作}: 区间加,区间乘

\verb|T = {mint s, l}| 区间和, 长度

\begin{minted}{cpp}
op(T a, T b) = return T{a.s + b.s, a.l + b.l}
e() = return T{0, 0}
F() = {mint mul, add}
mapping(F f, T a) = return T{a.s * f.mul + f.add * a.l, a.l}
composition(F f, F g) = return F{f.mul * g.mul, f.mul * g.add + f.add};
/*
g(x) = g.mul * x + g.add
f(g(x)) = f.mul * (g.mul * x + g.add) + f.add = f.mul * g.mul * x + f.mul * g.add + f.add
*/
id() = return F{1, 0}
\end{minted}
