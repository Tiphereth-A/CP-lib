区间加、区间和、区间最值

\begin{tabular}{ll}
    \hline
    代码                                    & 备注               \\
    \hline
    \verb|class T|                        & 维护的数据            \\
    \verb|T (*op)(T, T)|                  & 维护的操作            \\
    \verb|T (*e)()|                       & T的幺元             \\
    \verb|class F|                        & 修改的类型(会用于lazy标记) \\
    \verb|T (*mapping)(F, T) f(x)|        & 即修改怎么应用于数据       \\
    \verb|F (*composition)(F, F) f(g(x))| & 即修改的复合           \\
    \verb|F (*id)() id(x) = x|            & 即恒等映射            \\
    \hline
\end{tabular}

\paragraph{例子}

\textbf{没有区间操作}

\verb|F = T, maaping = op, id = composition = e|

\textbf{维护}:区间min;\textbf{操作}:区间加

\verb|T = {int _min}| 区间和,长度

\inputminted{cpp}{src/src/segtree_eg1.txt}

\textbf{维护}:区间和;\textbf{操作}:区间加、区间乘

\verb|T = {mint s, l}| 区间和,长度

\inputminted{cpp}{src/src/segtree_eg2.txt}
