使用改进的 Weiszfeld 算法, 也可以使用 \fullref{sec:模拟退火}

\paragraph{Weiszfeld 算法}

给定点集 \(X:=\{x_1,\dots,x_m\}\), 设结果为 \(y\),

\[
    \tilde{T}(y)=\left(\sum_{x_i\neq y} \frac{1}{\lVert y-x_i\rVert}\right)^{-1}\sum_{x_i\neq y} \frac{x_i}{\lVert y-x_i\rVert}.
\]

则 Weiszfeld 算法迭代格式为

\[
    y\gets T_0(y)=\begin{cases}
        \tilde{T}(y), &\text{if } y\notin\{x_1,\dots,x_m\},\\
        x_i,&\text{if } y=x_i.
    \end{cases}
\]

\paragraph{改进的 Weiszfeld 算法}

令 \(\eta(y)=[y\in\{x_1,\dots,x_m\}]\),

\[
    r(y)=\left\lVert\sum_{x_i\neq y} \frac{y-x_i}{\lVert y-x_i\rVert}\right\rVert.
\]

则改进的 Weiszfeld 算法迭代格式为

\[
    y\gets T(y)=\left(1-\frac{\eta(y)}{r(y)}\right)\tilde{T}(y)+\min\left\{1,\frac{\eta(y)}{r(y)}\right\}y
\]

当 \(y\notin\{x_1,\dots,x_m\}\) 时有 \(T(y)=T_0(y)=\tilde{T}(y)\)

\paragraph{参考文献} \cite{katz1974local} \cite{brimberg1992local} \cite{vardi2000multivariate}