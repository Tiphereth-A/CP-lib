区间修改,区间查询

\begin{tabular}{ll}
    \hline
    代码                                        & 备注                       \\
    \hline
    \verb|class T|                            & 维护的数据                    \\
    \verb|T (*op)(T, T)|                      & 维护的操作                    \\
    \verb|class F|                            & 修改的类型(会用于lazy标记)         \\
    \verb|void (*mapping)(T&, F) f(x)|        & 即修改怎么应用于数据. 结果直接作用于第一个参数 \\
    \verb|void (*composition)(F&, F) f(g(x))| & 即修改的复合. 结果直接作用于第一个参数     \\
    \verb|T e|                                & T的幺元                     \\
    \verb|F id; id(x) = x|                    & 即恒等映射                    \\
    \hline
\end{tabular}

不需要lazy标记的时候用 \verb|segtree_notag|, 此时 \verb|T = F|, \verb|mapping = composition| 且 \verb|mapping(x,y)| 的效果应等价于 |x = op(x,y)|

\paragraph{例子}

另请参阅 \fullref{sec:常用线段树构造}

\textbf{维护}: 区间min; \textbf{操作}: 区间加

\verb|T = {int _min}| 区间和, 长度

\begin{minted}{cpp}
op(T a, T b) -> T{min(a._min, b._min)}
F = {int add}
mapping(T& a, F f) -> a = T{a.s + f.add * a.l, a.l}
composition(F& f, F g) -> f = F{f.add + g.add}
e = T{INT_MAX}
id = F{0}
\end{minted}

\textbf{维护}: 区间和; \textbf{操作}: 区间加,区间乘

\verb|T = {mint s, l}| 区间和, 长度

\begin{minted}{cpp}
op(T a, T b) -> T{a.s + b.s, a.l + b.l}
F = {mint mul, add}
mapping(T& a, F f) -> a = T{a.s * f.mul + f.add * a.l, a.l}
composition(F& f, F g) -> f = F{f.mul * g.mul, f.mul * g.add + f.add};
// g(x) = g.mul * x + g.add
// f(g(x)) = f.mul * (g.mul * x + g.add) + f.add = f.mul * g.mul * x + f.mul * g.add + f.add
e = T{0, 0}
id = F{1, 0}
\end{minted}
