\paragraph{定义}

\textbf{根向树形图}: 基图是树, 且所有的边全部指向父亲的有向图

\textbf{叶向树形图}: 基图是树, 且所有的边全部指向儿子的有向图

记图 \(G\) 以 \(r\) 为根, 包含所有点的根向树形子图个数为 \(t^{root}(G,r)\)

记图 \(G\) 以 \(r\) 为根, 包含所有点的叶向树形子图个数为 \(t^{leaf}(G,r)\)

记图 \(G\) 以 \(r\) 为根的所有生成树个数为 \(t(G,r)\)

\paragraph{输入}

\verb|g|: \textbf{无权} 的邻接矩阵, \verb|g.g[i][j]| 表示从 \(i\) 到 \(j\) 的弧的条数, 忽略所有自环

\verb|r|: 根, 意义参见输出

\paragraph{输出}

对有向图来说, 若 \verb|outer == true| 则返回 \(t^{root}(G,r)\), 否则返回 \(t^{leaf}(G,r)\)

对无向图来说, 我们把边视作两条方向相反的弧, 所以无论 \verb|outer| 为何值, 返回的均是 \(t(G,r)\)

且无论 \(r\) 为何值, 结果总是相等的

\paragraph{BEST 定理}

设 \(G\) 是 \textbf{有向欧拉图}, 那么 \(G\) 的不同欧拉回路总数是

\[
    t^{root}(G,k)\prod_{v\in V}(\deg (v) - 1)!
\]

对欧拉图 \(G\) 的任意两点 \(k, k'\), 都有 \(t^{root}(G,k)=t^{root}(G,k')\), 且欧拉图 \(G\) 所有节点的入度和出度相等

可用 \fullref{sec:常用公式} 中的 LGV 引理证明

示例代码为 BEST 定理模板题