求解如下线性规划问题:

\[
    \begin{array}{ll}
        \max          & c^T x.             \\
        \textrm{s.t.} & Ax \le b, x \ge 0.
    \end{array}
\]

若无解则返回 \verb|-inf|, 若无界则返回 \verb|inf|, 否则返回 \(c^T x\) 的最大值. 输入的 \verb|x| 会被修改为解. 不保证数值稳定性. \(x = 0\) 最好在可行域中.

\paragraph{使用}

\begin{minted}{cpp}
using T = double;
vvec<T> A{{1, -1}, {-1, 1}, {-1, -2}};
vec<T> b{1, 1, -4}, c{-1, -1}, x;
T val = LPSolver(A, b, c).solve(x);
\end{minted}

\paragraph{原问题与对偶问题}~\\

\begin{tabular}{c|c}
    \hline
    原问题                               & 对偶问题                                \\
    \hline
    \(\max_x c^T x\)                     & \(\min_y b^T y\)                        \\
    \hline
    \(n\) 个变量 \(x_1,\dots,x_n\)       & \(n\) 个限制条件 \(A^T y\gtreqqless c\) \\
    \(x_i\geq 0\)                        & 第 \(i\) 个限制条件为 \(\geq c_i\)      \\
    \(x_i\leq 0\)                        & 第 \(i\) 个限制条件为 \(\leq c_i\)      \\
    \(x_i\in \mathrm{R}\)                & 第 \(i\) 个限制条件为 \(=c_i\)          \\
    \hline
    \(m\) 个限制条件 \(Ax\gtreqqless b\) & \(m\) 个变量 \(y_1,\dots,y_m\)          \\
    第 \(i\) 个限制条件为 \(\geq b_i\)   & \(y_i\leq 0\)                           \\
    第 \(i\) 个限制条件为 \(\leq b_i\)   & \(y_i\geq 0\)                           \\
    第 \(i\) 个限制条件为 \(=b_i\)       & \(y_i\in \mathrm{R}\)                   \\
    \hline
\end{tabular}

\begin{tabular}{cc|cc}
    \hline
    \multicolumn{2}{c}{原问题} & \multicolumn{2}{c}{对偶问题}                                            \\
    \hline
    \(\max\)                     & \(3x_1+4x_2\)                  & \(\min\)          & \(7y_1\)             \\
    \(\textrm{s.t.}\)            & \(5x_1+6x_2=7\)                & \(\textrm{s.t.}\) & \(5y_1\geq 3\)       \\
                                 &                                &                   & \(6y_2\geq 4\)       \\
                                 & \(x_1\geq 0,x_2\geq 0\)        &                   & \(y_1\in\mathrm{R}\) \\
    \hline
\end{tabular}

\paragraph{复杂度}

\(O(NM \cdot \texttt{pivots})\), where a pivot may be e.g. an edge relaxation. \(O(2^N)\) in the general case.

\paragraph{参考} \cite{bqi343cp}
