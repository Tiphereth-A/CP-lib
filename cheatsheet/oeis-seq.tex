From \cite{antileafstandard}.

如果没有特殊说明, 那么以下数列都从第 \(0\) 项开始, 除非没有定义也没有好的办法解释第 \(0\) 项的意义.

\subsection{计数相关}

\begin{enumerate}
    \item \textbf{Catalan 数(A000108)}

          1, 1, 2, 5, 14, 42, 132, 429, 1430, 4862, 16796, 58786, 208012, 742900, 2674440, 9694845, 35357670, \dots

          性质见 \fullref{sec:catalan-数}.

    \item \textbf{(大)Schr\"oder 数(A006318)}

          1, 2, 6, 22, 90, 394, 1806, 8558, 41586, 206098, 1037718, 5293446, 27297738, 142078746, 745387038, \dots

          性质见 \fullref{sec:schroder-数}.

    \item \textbf{小 Schr\"oder 数(A001003)}

          1, 1, 3, 11, 45, 197, 903, 4279, 20793, 103049, 518859, 2646723, 13648869, 71039373, 372693519, \dots

          性质见 \fullref{sec:schroder-数}.

          小 Schr\"oder 数除了第 \(0\) 项以外都是 Schr\"oder 数的一半.

    \item \textbf{Motzkin 数(A001006)}

          1, 1, 2, 4, 9, 21, 51, 127, 323, 835, 2188, 5798, 15511, 41835, 113634, 310572, 853467, 2356779, \dots

          性质见 \fullref{sec:motzkin-数}.

    \item \textbf{将点按顺序排成一圈后不自交的树的个数(A001764)}

          1, 1, 3, 12, 55, 273, 1428, 7752, 43263, 246675, 1430715, 8414640, 50067108, 300830572, 1822766520, \dots

          \[
              a_n = \frac {{3n \choose n}} {2n + 1}
          \]

          也就是说, 在圆上按顺序排列的 \(n\) 个点之间连 \(n - 1\) 条不相交(除端点外)的弦, 组成一棵树的方案数.

          也等于每次只能向右或向上, 并且不能高于 \(y = 2x\) 这条直线, 从 \((0, 0)\) 走到 \((n, 2n)\) 的方案数.

          扩展: 如果改成不能高于 \(y = kx\) 这条直线, 走到 \((n, kn)\) 的方案数, 那么答案就是 \( \frac {{(k + 1)n \choose n}} {kn + 1} \).

    \item \textbf{\(n\) 个点的圆上画不相交的弦的方案数(A054726)}

          1, 1, 2, 8, 48, 352, 2880, 25216, 231168, 2190848, 21292032, 211044352, 2125246464, 21681954816, \dots

          \( a_n = 2^n s_{n - 2} \; (n > 2) \), \(s_n\) 是上面的小 Schr\"oder 数.

          和上面的区别在于, 这里可以不连满 \(n-1\) 条边. 另外 Motzkin 数画的弦不能共享端点, 但是这里可以.

    \item \textbf{Wedderburn-Etherington numbers(A001190)}

          0, 1, 1, 1, 2, 3, 6, 11, 23, 46, 98, 207, 451, 983, 2179, 4850, 10905, 24631, 56011, 127912, 293547, \dots

          每个结点都有 \(0\) 或者 \(2\) 个儿子, 且总共有\(n\)个叶子结点的二叉树方案数. (\textbf{无标号})

          同时也是 \(n-1\) 个结点的\textbf{无标号}二叉树个数.

          \[
              A(x) = x + \frac {A(x) ^ 2 + A(x ^ 2)} 2 = 1 - \sqrt{1 - 2x - A(x ^ 2)}
          \]

    \item \textbf{划分数(A000041)}

          1, 1, 2, 3, 5, 7, 11, 15, 22, 30, 42, 56, 77, 101, 135, 176, 231, 297, 385, 490, 627, 792, 1002, \dots

    \item \textbf{Bell 数(A000110)}

          1, 1, 2, 5, 15, 52, 203, 877, 4140, 21147, 115975, 678570, 4213597, 27644437, 190899322, 1382958545, \dots

    \item \textbf{错位排列数(A0000166)}

          1, 0, 1, 2, 9, 44, 265, 1854, 14833, 133496, 1334961, 14684570, 176214841, 2290792932, 32071101049, \dots

    \item \textbf{交替阶乘(A005165)}

          0, 1, 1, 5, 19, 101, 619, 4421, 35899, 326981, 3301819, 36614981, 442386619, 5784634181, 81393657019, \dots

          \[
              \begin{aligned} n! - (n - 1)! + (n - 2)! - \dots 1! = \sum_{i = 0} ^ {n - 1} (-1)^i (n - i)! \end{aligned}
          \].

          \( a_0 = 0,\; a_n = n! - a_{n - 1} \).
\end{enumerate}

\subsection{线性递推数列}

\begin{enumerate}
    \item \textbf{Lucas 数(A000032)}

          2, 1, 3, 4, 7, 11, 18, 29, 47, 76, 123, 199, 322, 521, 843, 1364, 2207, 3571, 5778, 9349, 15127, \dots

    \item \textbf{Fibonacci 数(A000045)}

          0, 1, 1, 2, 3, 5, 8, 13, 21, 34, 55, 89, 144, 233, 377, 610, 987, 1597, 2584, 4181, 6765, 10946, \dots

    \item \textbf{Tribonacci 数(A000071)}

          0, 0, 1, 1, 2, 4, 7, 13, 24, 44, 81, 149, 274, 504, 927, 1705, 3136, 5768, 10609, 19513, 35890, \dots

          \( a_0 = a_1 = 0,\; a_2 = 1,\; a_n = a_{n - 1} + a_{n - 2} + a_{n - 3} \).

    \item \textbf{Tetranacci 数(A000078)}

          0, 0, 0, 1, 1, 2, 4, 8, 15, 29, 56, 108, 208, 401, 773, 1490, 2872, 5536, 10671, 20569, 39648, 76424, \dots

          \( a_0 = a_1 = a_2 = 0,\; a_3 = 1,\; a_n = a_{n - 1} + a_{n - 2} + a_{n - 3} + a_{n - 4} \).

    \item \textbf{Pell 数 / 2-Fibonacci 数(A0000129)}

          0, 1, 2, 5, 12, 29, 70, 169, 408, 985, 2378, 5741, 13860, 33461, 80782, 195025, 470832, 1136689, \dots

          \( a_0 = 0,\; a_1 = 1,\; a_n = 2a_{n - 1} + a_{n - 2} \).

    \item \textbf{3-Fibonacci 数(A006190)}

          0, 1, 3, 10, 33, 109, 360, 1189, 3927, 12970, 42837, 141481, 467280, 1543321, 5097243, 16835050, 55602393, \dots

          \( a_0 = 0,\; a_1 = 1,\; a_n = 3a_{n - 1} + a_{n - 2} \).

    \item \textbf{Padovan 数(A0000931)}

          1, 0, 0, 1, 0, 1, 1, 1, 2, 2, 3, 4, 5, 7, 9, 12, 16, 21, 28, 37, 49, 65, 86, 114, 151, 200, 265, 351, 465, 616, 816, 1081, 1432, 1897, 2513, 3329, 4410, 5842, 7739, 10252, 13581, 17991, 23833, 31572, \dots

          \(a_0 = 1,\; a_1 = a_2 = 0,\; a_n = a_{n - 2} + a_{n - 3}\).

    \item \textbf{Jacobsthal numbers / (1,2)-Fibonacci 数(A001045)}

          0, 1, 1, 3, 5, 11, 21, 43, 85, 171, 341, 683, 1365, 2731, 5461, 10923, 21845, 43691, 87381, 174763, \dots

          \( a_0 = 0,\; a_1 = 1.\; a_n = a_{n - 1} + 2a_{n - 2} \)

          同时也是最接近\(\frac {2 ^ n} 3\)的整数.

    \item \textbf{(1,3)-Fibonacci 数(A006130)}

          1, 1, 4, 7, 19, 40, 97, 217, 508, 1159, 2683, 6160, 14209, 32689, 75316, 173383, 399331, 919480, 2117473, 4875913, 11228332, 25856071, \dots

          \( a_0 = 0,\; a_1 = 1.\; a_n = a_{n - 1} + 3a_{n - 2} \)

    \item \textbf{Perrin 数(A001608)}

          3, 0, 2, 3, 2, 5, 5, 7, 10, 12, 17, 22, 29, 39, 51, 68, 90, 119, 158, 209, 277, 367, 486, 644, 853, \dots

          \( a_0 = 3,\; a_1 = 0,\; a_2 = 2,\; a_n = a_{n - 2} + a_{n - 3} \)
\end{enumerate}

\subsection{数论相关}

\begin{enumerate}
    \item \textbf{Carmichael 数, 伪质数(A002997)}

          561, 1105, 1729, 2465, 2821, 6601, 8911, 10585, 15841, 29341, 41041, 46657, 52633, 62745, 63973, 75361, 101101, 115921, 126217, 162401, 172081, 188461, 252601, 278545, 294409, 314821, 334153, 340561, 399001, 410041, 449065, 488881, 512461, \dots

          满足 \(\forall\) 与 \(n\) 互质的 \(a\), 都有 \(a ^ {n - 1} \equiv 1 \pmod n\) 的所有\textbf{合数} \(n\) 被称为 Carmichael 数.

          Carmichael数在 \(10^8\) 以内只有255个.

    \item \textbf{反质数(A002182)}

          1, 2, 4, 6, 12, 24, 36, 48, 60, 120, 180, 240, 360, 720, 840, 1260, 1680, 2520, 5040, 7560, 10080, 15120, 20160, 25200, 27720, 45360, 50400, 55440, 83160, 110880, 166320, 221760, 277200, 332640, 498960, 554400, 665280, 720720, 1081080, 1441440, 2162160, \dots

          比所有更小的数的约数数量都更多的数.

    \item \textbf{前 \(n\) 个质数的乘积(A002110)}

          1, 2, 6, 30, 210, 2310, 30030, 510510, 9699690, 223092870, 6469693230, 200560490130,\\7420738134810, \dots

    \item \textbf{Mersenne 质数(A000668)}

          3, 7, 31, 127, 8191, 131071, 524287, 2147483647, 2305843009213693951,\\618970019642690137449562111, 162259276829213363391578010288127,\\170141183460469231731687303715884105727

          \(p\) 是质数, 同时 \(2^p - 1\) 也是质数.

    \item \textbf{Pisano 周期}: 模 \(n\) 的最小周期.

          \textbf{Fibonacci 数(A001175)}: 1, 3, 8, 6, 20, 24, 16, 12, 24, 60, 10, 24, 28, 48, 40, 24, 36, 24, 18, 60, 16, 30, 48, 24, 100, 84, 72, 48, 14, 120, 30, 48, 40, 36, 80, 24, 76, 18, 56, 60, 40, 48, 88, 30, 120, 48, 32, 24, 112, 300, 72, 84, 108, 72, 20, 48, 72, 42, 58, 120, 60, 30, 48, 96, 140, 120, 136, \dots

          参见 \fullref{sec:模-fibonacci-循环节}

          \textbf{Tribonacci 数(A046738)}: 1, 4, 13, 8, 31, 52, 48, 16, 39, 124, 110, 104, 168, 48, 403, 32, 96, 156, 360, 248, 624, 220, 553, 208, 155, 168, 117, 48, 140, 1612, 331, 64, 1430, 96, 1488, 312, 469, 360, 2184, 496, 560, 624, 308, 440, 1209, 2212, 46, 416, 336, 620, 1248, 168, \dots

          \textbf{Tetranacci 数(A106295)}: 1, 5, 26, 10, 312, 130, 342, 20, 78, 1560, 120, 130, 84, 1710, 312, 40, 4912, 390, 6858, 1560, 4446, 120, 12166, 260, 1560, 420, 234, 1710, 280, 1560, 61568, 80, 1560, 24560, 17784, 390, 1368, 34290, 1092, 1560, 240, 22230, 162800, 120, 312, 60830, 103822, \dots

          也是初值为 \(a_0=4, a_1=1, a_2=2, a_3=7\) 的 Tetranacci 数列的 Pisano 周期

          \textbf{Lucas 数(A106291)}: 1, 3, 8, 6, 4, 24, 16, 12, 24, 12, 10, 24, 28, 48, 8, 24, 36, 24, 18, 12, 16, 30, 48, 24, 20, 84, 72, 48, 14, 24, 30, 48, 40, 36, 16, 24, 76, 18, 56, 12, 40, 48, 88, 30, 24, 48, 32, 24, 112, 60, 72, 84, 108, 72, 20, 48, 72, 42, 58, 24, 60, 30, 48, 96, 28, 120, 136, 36, 48, 48, \dots

          \textbf{Pell 数 / 2-Fibonacci 数(A175181)}: 1, 2, 8, 4, 12, 8, 6, 8, 24, 12, 24, 8, 28, 6, 24, 16, 16, 24, 40, 12, 24, 24, 22, 8, 60, 28, 72, 12, 20, 24, 30, 32, 24, 16, 12, 24, 76, 40, 56, 24, 10, 24, 88, 24, 24, 22, 46, 16, 42, 60, 16, 28, 108, 72, 24, 24, 40, 20, 40, 24, 124, 30, 24, 64, 84, 24, 136, \dots

          \textbf{3-Fibonacci 数(A175182)}: 1, 3, 2, 6, 12, 6, 16, 12, 6, 12, 8, 6, 52, 48, 12, 24, 16, 6, 40, 12, 16, 24, 22, 12, 60, 156, 18, 48, 28, 12, 64, 48, 8, 48, 48, 6, 76, 120, 52, 12, 28, 48, 42, 24, 12, 66, 96, 24, 112, 60, 16, 156, 26, 18, 24, 48, 40, 84, 24, 12, 30, 192, 48, 96, 156, 24, 136, 48, 22, 48, 144, \dots

          \textbf{Jacobsthal numbers / (1,2)-Fibonacci 数(A175286)}: 1, 1, 6, 2, 4, 6, 6, 2, 18, 4, 10, 6, 12, 6, 12, 2, 8, 18, 18, 4, 6, 10, 22, 6, 20, 12, 54, 6, 28, 12, 10, 2, 30, 8, 12, 18, 36, 18, 12, 4, 20, 6, 14, 10, 36, 22, 46, 6, 42, 20, 24, 12, 52, 54, 20, 6, 18, 28, 58, 12, 60, 10, 18, 2, 12, 30, 66, 8, 66, 12, 70, 18, 18, 36, 60, 18, 30, 12, 78, 4, \dots

          \textbf{(1,3)-Fibonacci 数(A175291)}: 1, 3, 1, 6, 24, 3, 24, 6, 3, 24, 120, 6, 156, 24, 24, 12, 16, 3, 90, 24, 24, 120, 22, 6, 120, 156, 9, 24, 28, 24, 240, 24, 120, 48, 24, 6, 171, 90, 156, 24, 336, 24, 42, 120, 24, 66, 736, 12, 168, 120, 16, 156, 52, 9, 120, 24, 90, 84, 3480, 24, 20, 240, 24, 48, 312, 120, 748, 48, 22, 24, 5040, 6, 888, 171, 120, 90, 120, 156, 39, \dots
\end{enumerate}

\subsection{其他}

\begin{enumerate}
    \item \textbf{Bernoulli 数(A027641)}

          见 \fullref{sec:bernoulli-数与自然数幂和}.

    \item \textbf{四个柱子的汉诺塔(A007664)}

          0, 1, 3, 5, 9, 13, 17, 25, 33, 41, 49, 65, 81, 97, 113, 129, 161, 193, 225, 257, 289, 321, 385, 449, \dots

          差分之后可以发现其实就是 1 次 +1, 2 次 +2, 3 次 +4, 4 次 +8\dots 的规律.

    \item \textbf{Ulam 数(A002858)}

          1, 2, 3, 4, 6, 8, 11, 13, 16, 18, 26, 28, 36, 38, 47, 48, 53, 57, 62, 69, 72, 77, 82, 87, 97, 99, 102, 106, 114, 126, 131, 138, 145, 148, 155, 175, 177, 180, 182, 189, 197, 206, 209, 219, 221, 236, 238, 241, 243, 253, 258, 260, 273, 282, 309, 316, 319, 324, 339, \dots

          \( a_1 = 1,\; a_2 = 2 \), \(a_n\) 表示在所有 \(>a_{n-1}\) 的数中, 最小的, 能被表示成 (前面的两个不同的元素的和) 的数.
\end{enumerate}
