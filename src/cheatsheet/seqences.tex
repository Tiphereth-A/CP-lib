From \cite{antileafstandard}.

另请参阅 \fullref{sec:常用-oeis-数列}

\subsection{Fibonacci 数与 Lucas 数}
\label{sec:fibonacci-数与-lucas-数}

\subsubsection{Fibonacci 数}
\label{ssec:fibonacci-数}

\[
    F_0 = 0, \, F_1 = 1, \, F_n = F_{n - 1} + F_{n - 2}
\]

\subsubsection{Lucas 数}
\label{ssec:lucas-数}

\[
    L_0 = 2, \, L_1 = 1, \, L_n = L_{n - 1} + L_{n - 2}
\]

\subsubsection{通项公式}

\[
    F_n = \frac{\phi^n - {\hat\phi}^n}{\sqrt 5}, \, L_n = \phi^n + {\hat\phi}^n
\]

其中 \(\phi = \dfrac{1 + \sqrt 5}{2}, \, \hat\phi = \dfrac{1 - \sqrt 5}{2}\)

实际上有 \(\dfrac{L_n + F_n \sqrt 5}{2} = \left( \dfrac{1 + \sqrt 5}{2} \right)^n\), 所以求通项的话写一个类然后快速幂就可以同时得到两者.

\subsubsection{快速倍增法}

\[
    F_{2k} = F_k \left( 2 F_{k + 1} - F_k \right), \, F_{2k + 1} = F_{k + 1}^2 + F_k^2
\]

\inputminted{cpp}{src/src/fib.cpp}

\subsection{Bernoulli 数与自然数幂和}
\label{sec:bernoulli-数与自然数幂和}

\subsubsection{EGF}

\[
    B(x)=\sum_{i\ge 0}\frac{B_i x^i}{i!}=\frac x{\mathrm{e}^x-1}
\]

\[
    B_n=[n=0]-\sum_{i=0}^{n-1}\binom{n}{i}\frac{B_i}{n-k+1}
\]

\[
    \sum_{i=0}^n\binom{n+1}{i}B_i=0
\]

\[
    S_n(m)=\sum_{i=0}^{m-1}i^n=\sum_{i=0}^n\binom{n}{i}B_{n-i}\frac{m^{i+1}}{i+1}
\]

\[
    B_0 = 1, \, B_1 = -\frac{1}{2}, \, B_4 = -\frac{1}{30}, \, B_6 = \frac{1}{42}, \, B_8 = -\frac{1}{30}, \, \dots
\]

除了 \(B_1 = -\dfrac{1}{2}\) 以外, Bernoulli 数的奇数项都是 \(0\).

自然数幂次和关于次数的 EGF:

\[
    F(x)=\sum_{k=0}^\infty \frac{\sum_{i=0}^n i^k}{k!}x^k=\sum_{i=0}^n \mathrm{e}^{ix}=\frac{\mathrm{e}^{(n+1)x-1}}{\mathrm{e}^x-1}
\]

\subsection{Stirling 数}
\label{sec:stirling-数}

\subsubsection{第一类 Stirling 数}
\label{ssec:第一类-stirling-数}

\(n\brack k\) 表示 \(n\) 个元素划分成 \(k\) 个轮换的方案数.

递推式:

\[
    \stirlingI{n}{k} = \stirlingI{n-1}{k-1} + (n-1)\stirlingI{n-1}{k}
\]

求同一行: 分治 FFT \(O(n\log ^2 n)\), 或者倍增 \(O(n\log n)\) \\(每次都是 \(f(x) = g(x) g(x + d)\) 的形式, 可以用 \(g(x)\) 反转之后做一个卷积求出后者).

\[
    \sum_{k = 0}^n \stirlingI{n}{k} x^k = \prod_{i = 0}^{n - 1} (x + i)
\]

求同一列: 用一个轮换的指数生成函数做 \(k\) 次幂

\[
    \sum_{n = 0}^\infty \stirlingI{n}{k} \frac{x^n}{n!} = \frac{\left(\ln (1 - x)\right)^k}{k!} = \frac{x^k}{k!} \left( \frac{\ln (1 - x)}{x} \right)^k
\]

\subsubsection{第二类 Stirling 数}
\label{ssec:第二类-stirling-数}

\(n\brace k\) 表示 \(n\) 个元素划分成 \(k\) 个子集的方案数.

递推式:

\[
    \stirlingII{n}{k} = \stirlingII{n-1}{k-1} + k\stirlingII{n-1}{k}
\]

求一个: 容斥

\[
    \stirlingII{n}{k} = \frac{1}{k!} \sum_{i = 0}^k (-1)^i \binom{k}{i} (k - i)^n = \sum_{i = 0}^k \frac{(-1)^i}{i!} \frac{(k - i)^n}{(k - i)!}
\]

求同一行: FFT

求同一列: 指数生成函数

\[
    \sum_{n = 0}^\infty \stirlingII{n}{k} \frac{x^n}{n!} = \frac{\left(\mathrm{e}^x - 1\right)^k}{k!} = \frac{x^k}{k!} \left( \frac{\mathrm{e}^x - 1}{x} \right)^k
\]

普通生成函数

\[
    \sum_{n = 0}^\infty \stirlingII{n}{k} x^n = x^k \left(\prod_{i = 1}^k (1 - i x)\right)^{-1}
\]

\subsubsection{Stirling 反演}

\[
    f(n) = \sum_{k = 0}^n \stirlingII{n}{k} g(k) \iff g(n) = \sum_{k = 0}^n (-1)^{n - k} \stirlingI{n}{k} f(k)
\]

\subsubsection{幂的转换}

\textbf{上升幂与普通幂的转换}

\[
    x^{\overline{n}}=\sum_{k} \stirlingI{n}{k} x^k
\]

\[
    x^n=\sum_{k} \stirlingII{n}{k} (-1)^{n-k} x^{\overline{k}}
\]

\textbf{下降幂与普通幂的转换}

\[
    x^n=\sum_{k} \stirlingII{n}{k} x^{\underline{k}} = \sum_{k} \binom{x}{k} \stirlingII{n}{k} k!
\]

\[
    x^{\underline{n}}=\sum_{k} \stirlingI{n}{k} (-1)^{n-k} x^k
\]

多项式的\textbf{点值}表示的每项除以阶乘之后卷上 \(\mathrm{e}^{-x}\) 乘上阶乘之后是牛顿插值表示, 不乘阶乘就是\textbf{下降幂}系数表示.

反过来的转换当然卷上 \(\mathrm{e}^x\) 就行了. 原理是每次差分等价于乘以 \((1 - x)\), 展开之后用一次卷积取代多次差分.

\subsubsection{Stirling 多项式(Stirling 数关于斜线的性质)}

定义:

\[
    \sigma_n(x) = \frac{\stirlingI{x}{n}}{x(x-1)\dots(x-n)}
\]

\(\sigma_n(x)\) 的最高次数是 \(x^{n - 1}\). (所以作为唯一的特例, \(\sigma_0(x) = \dfrac{1}{x}\) 不是多项式.)

Stirling 多项式实际上非常神奇, 它与两类 Stirling 数都有关系.

\[
    \stirlingI{n}{n-k} = n^{\underline{k+1}} \sigma_k(n)
\]

\[
    \stirlingII{n}{n-k} = (-1)^{k+1} n^{\underline{k+1}} \sigma_k(-(n-k))
\]

不过它并不好求. 可以 \(O(k^2)\) 直接计算前几个点值然后插值, 或者如果要推式子的话可以用 \fullref{ssec:二阶-eulerian-number}.

\subsection{Bell 数}
\label{sec:bell-数}

\[
    B_0 = 1,\, B_1 = 1,\, B_2 = 2,\, B_3 = 5
\]

\[
    B_4 = 15,\, B_5 = 52,\, B_6 = 203, \dots
\]

\[
    B_n = \sum_{k = 0} ^ n \stirlingII{n}{k}
\]

\subsubsection{递推式}

\[
    B_{n + 1} = \sum_{k = 0} ^n \binom{n}{k} B_k
\]

\subsubsection{指数生成函数}

\[
    B(x) = \mathrm{e}^{\mathrm{e}^x - 1}
\]

\subsubsection{Touchard 同余}

\[
    B_{n + p} \equiv (B_n + B_{n + 1}) \pmod p,~p\in\mathbb{P}
\]

\subsection{Eulerian Number}
\label{sec:eulerian-number}

\subsubsection{Eulerian Number}
\label{ssec:eulerian-number}

\(\eulerian{n}{k}\): \(n\) 个数的排列, 有 \(k\) 个上升的方案数.

\[
    \eulerian{n}{k} = (n - k)\eulerian{n - 1}{k - 1} + (k + 1)\eulerian{n - 1}{k}
\]

\[
    \eulerian{n}{k} = \sum_{i = 0} ^ {k + 1} (-1)^i \binom{n + 1}{i} (k + 1 - i)^n
\]

\[
    \sum_{k = 0} ^ {n - 1} \eulerian{n}{k} = n!
\]

\[
    x^n = \sum_{k = 0} ^ {n - 1} \eulerian{n}{k} \binom{x+k}{n}
\]

\[
    k!\stirlingII{n}{k} = \sum_{i = 0} ^ {n - 1} \eulerian{n}{i} \binom{i}{n - k}
\]

\subsubsection{二阶 Eulerian Number}
\label{ssec:二阶-eulerian-number}

\(\eulerianII{n}{k}\): 每个数都出现两次的多重排列, 并且每个数两次出现之间的数都比它要大. 在此前提下有 \(k\) 个上升的方案数.

\[
    \eulerianII{n}{k} = (2n-k-1)\eulerianII{n-1}{k-1} + (k+1)\eulerianII{n-1}{k}
\]

\[
    \sum_{k = 0} ^ {n - 1} \eulerianII{n}{k} = (2n-1)!! = \frac{(2n)^{\underline n}} {2^n}
\]

\subsubsection{二阶 Eulerian Number 与 Stirling 数的关系}
\label{ssec:二阶-eulerian-number-与-Stirling-数的关系}

\[
    \stirlingII{x}{x-n} = \sum_{k = 0} ^ {n - 1} \eulerianII{n}{k} \binom{x + n - k - 1}{2n}
\]

\[
    \stirlingI{x}{x-n} = \sum_{k = 0} ^ {n - 1} \eulerianII{n}{k} \binom{x + k}{2n}
\]

\subsection{Catalan 数}
\label{sec:catalan-数}

\[
    C_n = \frac{1}{n + 1}\binom{2n}{n} = \binom{2n}{n} - \binom{2n}{n - 1}
\]

\begin{itemize}
    \item \(n\) 个元素按顺序入栈, 出栈序列方案数
    \item 长为 \(2n\) 的合法括号序列数
    \item \(n + 1\) 个叶子的满二叉树个数
\end{itemize}

递推式: 

\[
    C_n = \sum_{i = 0} ^ {n - 1} C_i C_{n - i - 1}
\]

\[
    C_n = C_{n - 1} \frac {4n - 2} {n + 1}
\]

普通生成函数:

\[
    C(x) = \frac{1 - \sqrt {1 - 4 x}}{2 x}
\]

扩展: 如果有 \(n\) 个左括号和 \(m\) 个右括号, 方案数为

\[
    \binom{n + m}{n} - \binom{n + m}{m - 1}
\]

\subsection{Schr\"oder 数}
\label{sec:schroder-数}

\[
    S_n = S_{n-1} + \sum_{i = 0} ^ {n - 1} S_i S_{n - i - 1}
\]

\[
    (n + 1)s_n = (6n - 3)s_{n - 1} - (n - 2) s_{n - 2}
\]

其中 \(S_n\) 是(大) Schr\"oder 数, \(s_n\) 是小 Schr\"oder 数 (也叫超级 Catalan 数).

除了 \(S_0 = s_0 = 1\) 以外, 都有 \(S_i = 2s_i\).

Schr\"oder 数的组合意义:

\begin{itemize}
    \item 从 \((0, 0)\) 走到 \((n, n)\), 每次可以走右, 上, 或者右上一步, 并且不能超过 \(y=x\) 这条线的方案数
    \item 长为 \(n\) 的括号序列, 每个位置也可以为空, 并且括号对数和空位置数加起来等于 \(n\) 的方案数
    \item 凸 \(n\) 边形的\textbf{任意}剖分方案数
\end{itemize}

有些人会把大(而不是小) Schr\"oder 数叫做超级 Catalan 数.

\subsection{Motzkin 数}
\label{sec:motzkin-数}

\[
    M_{n + 1} = M_n + \sum_{i = 0} ^ {n - 1} M_i M_{n - 1 - i} = \frac{(2n + 3)M_n + 3n M_{n - 1}}{n + 3}
\]

\[
    M_n = \sum_{i = 0} ^ {n/2} \binom{n}{2i} C_i
\]

在圆上的 \(n\) 个\textbf{不同的}点之间画任意条不相交(包括端点)的弦的方案数.

也等价于在网格图上, 每次可以走右上, 右下, 正右方一步, 且不能走到 \(y<0\) 的位置, 在此前提下从 \((0, 0)\) 走到 \((n, 0)\) 的方案数.

扩展: Motzkin 数画的弦不可以共享端点. 如果可以共享端点的话是A054726, 后面的表里可以查到.

